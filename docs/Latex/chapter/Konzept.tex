\section{Designkonzept}

% Klassenzimmer --> Flugschule um weiteren Komponenten sinnvoll einzufügen
%--> erstellung eines händischen drahtmodells -->
% --> Definition der einzelnen Komponenten --> Maaße definieren
% --> Visio Zeichnung 
% --> Interaktionen definieren
% Flugsimulator als erweiterung

%TODO Mention Phong from Flightsimulator

Als grundlegende Idee wurde zunächst ein Vorlesungsaal vorgeschlagen.
Um weitere Komponenten aus den Anforderungen an diese Arbeit sinnvoll umzusetzen,
wurde die grundlegende Idee überdacht und neu definiert als Klassenzimmer einer Flugschule.
\newparagraph
Um eine erste Vorstellung des Klassenzimmers zu bekommen wurde zunächst eine händische Zeichnung angefertigt.
Diese ist in der Abbildung \ref{fig:KlassenzimmerSkizze} dargestellt.
\begin{figure}[H]
  \centering
  \includegraphics[width=1\textwidth]{images/roomModel_OneNote.pdf}
  \caption{Klassenzimmer Skizze}
  \label{fig:KlassenzimmerSkizze}
\end{figure}\noindent
Anschließend wurde der Raum maßstabsgetreu in einem Bauplan gezeichnet um so die Abstände und Maaße teilweise zu definieren.
Diese Zeichnung ist in der Abbildung \ref{fig:KlassenzimmerEntwurf} dargestellt.

\begin{figure}[H]
  \centering
  \includegraphics[width=1\textwidth]{images/roomModelVisio.pdf}
  \caption{Klassenzimmer Entwurf mit Bemaßung}
  \label{fig:KlassenzimmerEntwurf}
\end{figure}\noindent
Um die Anforderungen vollständig zu erfüllen müssen Interaktionen mit der 3D Szene möglich sein, diese werden im Folgenden beschrieben.
\newparagraph
Im Klassenzimmer ist es möglich mit W A S D zu laufen, hierzu wird die Kamera auf einer Höhe durch den Raum bewegt, bei einer Kolision mit einem Gegenstand wird die Bewegung angehalten.
Das Licht im Klassenzimmer kann durch drei Lichtschalter neben der Tür per Mausclick gesteuert werden. Außerdem können die Stühle auf- und abgestuhlt werden. Bei einem Blick aus dem Fenster
soll die DHBW dargestellt werden, diese wird als \ac{HDR} Bild eingebunden. Zusätzlich kann man die Schränke öffnen und schließen und die Tafel hoch bzw. nach unten schieben.
\newparagraph
Aus den beschrieben Animationen und dem Plan aus Abbildung \ref{fig:KlassenzimmerEntwurf} ergeben sich alle Komponenten der Szene. Einige Maaße werden bereits durch
den Plan vorgegeben in einem weiteren Schritt werden diese nun vollständig definiert.
% Diese Definition sind in der Tabelle %\ref{} abgebildet
%//TODO Tabelle einfügen.
\newparagraph
Aus den oben definierten Größen können anschließend die Blender Modelle erstellt werden.
%//TODO FlugSim