\section{Tools}
Wie in der Einleitung erwähnt werden für den Programmentwurf die Programmiersprachen HMTL, CSS, JavaScript und WebGLv2 verwendet.
Die 3D Szenenmodellierung wird mit der three.js Bibliothek realisiert.
Zum Erstellen der 3D Modelle wird Blender verwendet. Diese können anschließend als
glTF (GL Transmission Format) in three.js importiert werden.
\newparagraph
Zur Entwicklung des Sourcecodes wird der Editor Visual Studio Code verwendet.
Um die aktuelle Website zu starten wird ein node-Server verwendet.
Der Sourcecode wird in einem Git Repository auf GitHub verwaltet.
Zur Dokumentation des Quellcodes wird JSDoc verwendet, dieses veranschaulicht sämtliche Funktionen und deren Kommentare in einer automatisch generierten Website.
Ergänzend dazu wird das Python Paket \href{https://github.com/scottrogowski/code2flow}{code2flow} verwendet, welches Flowcharts, die die Aufrufhierarchie der verwendeten Funktionen darstellt, erstellt.
\newparagraph
Um die Szene zu entwickeln und Entwürfe grafisch darzustellen wird Microsoft Visio verwendet.
Händische Zeichnungen werden mit Microsoft OneNote oder GoodNotes abhängig vom Teammitglied erstellt,
da GoodNotes nur auf Apple Geräten verfügbar ist.
\newparagraph
Alle verwendeten Hilfsmittel werden in der folgenden Auflistung dargestellt:
\begin{itemize}
  \item Blender
  \item code2flow
  \item GitHub
  \item GoodNotes
  \item JsDoc
  \item Microsoft OneNote
  \item Microsoft Visio
  \item Node Server
  \item Visual Studio Code
\end{itemize}